\section{Выводы}
В ходе выполнения лабораторной работы я освежил в памяти курс математической статистики: гистограмму, корелляцию и корреляционную матрицу 
для наборов данных. Так же я изучил библиотеку Pandas, она оказалась очень удобной для анализа данных.

Трудно было найти подходящий набор данных, который подходил бы под параметры для обучения линейных моделей. В ходе своих поисков я также пробовал
провести анализ и обучение на датасете для выявление диабета, но, по парным графикам все значения были \enquote{перемешаны} и корреляция почти 
всех признаков была  $\le 10\%$.

Был проанализирован набор данных Rice type \cite{kaggle}, результаты получились закономерные: тип риса напрямую зависит от геометрических параметров 
и региона выращивания. Но, исходя из корреляционной матрицы можно заметить, что рагион выращивания почти полностью определяет геометрические параметры
риса.
\pagebreak
